\section{Durchführung}
\label{sec:Durchführung}

Zunächst wird das Interferometer justiert.
Dazu wird der Laser eingeschaltet und die Spiegel werden so justiert, dass am Detektor die zwei hellsten Punkte übereinander liegen.
Anschließend wird dieser auf die entsprechende Höhe gebracht und eine Linse zur Vergrößerung des Interferenzmusters angebracht.
Ändert sich nun ein Intensitätsminimum zu einem Maximun wird an einem angeschlossenen Zähler hochgezählt.

Mittels einer motorbetriebene Mikrometerschraube kann nun die Distanz zu einem der Spiegel langsam variiert werden.
Diese wird so betrieben, dass sie einem Ausschlag zwischen 2 und 8 mm hat.
Der Motor wird nun laufen gelassen, bis der Zähler 3000 erreicht hat. Die genaue Zahl sowie der genaue Anfangs- und Endwert der Schraube werden notiert.
Diese Messung wird 10 mal wiederholt.

Anschließend wird Luft aus einem Behältnis vor dem Spiegel herausgepumpt und vorsichtig wieder herein gelassen.
Während des Befüllens wird der Zähler eingeschaltet und die Zahl wird notiert.
Auch diese Messung wird 10 mal wiederholt.