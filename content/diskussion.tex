\section{Diskussion}
\label{sec:Diskussion}
Die Wellenlänge des Lasers liegt bei 
\\ \\
    \centerline{$\overline{\lambda} = (633.52 \pm 1.86) \cdot 10^{-9} $ m.}
\\ \\
Die Abweichung durch die Mittelung ist mit $0.29 \%$ sehr gering, was für eine generell gute Messung spricht, bei der die Messwerte 
nur geringfügig voneinander abweichen. Dies weist auch auf eine geringe Auswirkung möglicher Fehlerquellen hin, wie die Verfälschung der 
Messung durch leichte Stöße am Tisch und damit am Versuchsaufbau die nicht immer zu vermeiden sind. Des Weiteren sind Fehler in der Elektronik, 
wie zum Beispiel der Diode möglich, welche aber als sehr gering einzuschätzen sind.
Im zweiten Versuchsteil ergab sich ein Brechungsindex von 
 \\ \\
    \centerline{$n = 1.0002756 \pm 0.0000008$}
\\ \\
für die Brechung in Luft. Bei der Messung trat zehnmal der gleiche Messwert auf, was für eine genaue Messung spricht, weshalb bei 
diesem Wert auch keine Abweichung vorliegt. 
Der Literaturwert für den Brechungsindex in Luft bei Normaldruck beträgt 
\\ \\
    \centerline{$n_{\symup{Lit}} = 1,000272 $ \cite{spektrum}}
\\ \\
und liegt damit $4.5 \%$ unterhalb des bestimmten Wertes. Diese Abweichung ist als gut einzuschätzen, da diese Messung einige Fehlerquellen 
aufweist. Der Fehler der zuvor bestimmten Wellenlänge ist gering und auch in dieser Rechung als gering einzustufen. Des Weiteren sind 
Fehlerquellen innerhalb der Messung vorhanden. Die Evakuierung ist durch manuelles Betätigen einer Pumpe vorzunehmen, wobei der Druck
von einer analogen Anzeige abzulesen ist und daher fehleranfällig ist. Außerdem ist die Geschwindigkeit der herrausströmenden Luft 
ebenfalls manuell geregelt und sorgt daher auch für Ungenauigkeiten.
