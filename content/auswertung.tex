\section{Auswertung}
\label{sec:Auswertung}
    \subsection{Bestimmung der Wellenlänge des Lasers}
        Zur Bestimmung der Wellenlänge des Lichtes des Lasers, werden aus den gemessenen Daten zu den Verschiebungen des Spiegels 
        $\Delta d$ und der zugehörigen Anzahl an Interferenzmaxima $z$ mittels Gleichung \eqref{eqn:verschieben} die jeweiligen 
        Wellenlängen $\lambda$ bestimmt. Zu Beachten ist hierbei allerdings die Hebelübersetzung $U$ am Spiegel, womit alle gemessenen Verschiebungen
        $\Delta d$ zunächst mit einem Faktor $\frac{1}{U}$ multipliziert werden müssen. Hier ist die Hebelübersetzung als $U = 5.017$ bekannt.
        Alle beschriebenen Werte sind in \autoref{tab:messung1} zu sehen.
        \begin{table}[!htp]
\centering
\caption{Gemessenene Verschiebungen mit eingerechneter Übersetzung und zugehöriger Anzahl Interferenzmaxima, sowie die jeweilige resultierende Wellenlänge.}
\label{tab:messung1}
\begin{tabular}{c c c c}
\toprule
{$\Delta d$ / $10^{-3}$ m} & {$\frac{\Delta d} {U} $ / $10^{-3}$ m} & {$z$} & {$\lambda$ $10^{-9}$ m} \\
\midrule
4.73 & 0.943 & 3009 & 626.65 \\
4.80 & 0.957 & 3002 & 637.41 \\
4.75 & 0.947 & 3001 & 630.98 \\
4.87 & 0.971 & 3006 & 645.84 \\
4.71 & 0.939 & 3001 & 625.66 \\
4.79 & 0.955 & 3006 & 635.23 \\
4.77 & 0.951 & 3000 & 633.84 \\
4.75 & 0.947 & 3000 & 631.19 \\
4.75 & 0.947 & 3001 & 630.98 \\
4.80 & 0.957 & 3002 & 637.41 \\
\bottomrule
\end{tabular}
\end{table}
        Die bestimmten Wellenlängen werden über 
        \begin{equation}
        \label{eqn:mittellung}
            \overline{\lambda} = \frac{1}{10} \sum_{i = 1}^{10} \lambda_{i}
        \end{equation}
        gemittelt, wobei $\lambda_{i}$ die einzelnen Wellenlängen sind, welche in \autoref{tab:messung1} zu sehen sind.
        Der Fehler dieses Mittelwertes lässt sich mittels 
        \begin{equation}
        \label{eqn:fehler_mittel}
            \Delta \overline{\lambda} = \frac{1}{\sqrt{90}} \sqrt{\sum_{i=1}^{10} (\lambda_i - \overline{\lambda})^2}
        \end{equation}    
        errechnen, womit sich die gemittelte Wellenlänge dann zu 
        \\ \\
        \centerline{$\overline{\lambda} = (633.52 \pm 1.86) \cdot 10^{-9} $ m}
        \\ \\
        ergibt.
    \subsection{Bestimmung des Brechungsindizes von Luft}
            Zur Bestimmung des Brechungsindizes von Luft wird bei einem ansteigenden Druck angefangen bei $0.4$ bar jeweils 10 mal 
            die Anzahl der Interferenzmaxima $z$ aufgenommen. Wie in \autoref{tab:messung2} zu sehen ist, ist hier 10 mal der gleiche 
            Messwert aufgenommen worden, weshalb die Betrachtung einer Abweichung hier unsinnig ist und der Wert von $z = 24$ als genau 
            angenommen wird.
            \begin{table}[!htp]
\centering
\caption{Gemessene Anzahl Interferenzmaxima bei konstantem Anfangsdruck $p' = 0.4$ bar.}
\label{tab:messung2}
\begin{tabular}{c}
\toprule
{$z$} \\
\midrule
24 \\
24 \\
24 \\
24 \\
24 \\
24 \\
24 \\
24 \\
24 \\
24 \\
\bottomrule
\end{tabular}
\end{table}
            Über Gleichung \eqref{eqn:brech} kann die Änderung des Brechungsindizes $\Delta n$ bestimmt werden, wobei
            $b = 50 \cdot 10^{-3}$ m die Größe der Messzelle und $\lambda$ die im vorherigen Auswertungsteil bestimmte Wellenlänge ist.
            Der Wert ergibt sich somit zu 
            \\ \\
            \centerline{$\Delta n = (0.1520 \pm 0.0004) \cdot 10^{-3}$,}
            \\ \\
            wobei der sich der Fehler über 
            \begin{equation}
               \symup{d} \Delta n = \frac{z}{2 \cdot b} \Delta \lambda
            \end{equation}
            berechnet. Hierbei ist $\Delta \lambda$ der Fehler der Wellenlänge.
            Der tatsächliche Brechungsindex $n$ lässt sich über ??? bestimmen, wobei $T_0 = 273.15$ K die Normaltemperatur, $p_0 = 1.032$ bar 
            der Normaldruck ist. Die Umgebungstemperatur beträgt $T = 293.15$ K. Des Weiteren ist die Messung bei Drücken von $p' = 0.4$ bar
            bis $p = 1$ bar durchgeführt worden. Damit ist der Brechungsindex
            \\ \\
            \centerline{$n = 1.0002756 \pm 0.0000008$,}
            \\ \\
            wobei sich der Fehler über
            \begin{equation}
                \symup{d} n = \frac{T}{T_0} \frac{p_0}{p- p'} \cdot d \Delta n
            \end{equation}
            errechnet.    